\section{The emulator} \label{sec:emu}

For a given combination of linear power parameters $\theta$, i.e., for
each point in the lookup table, we want to compute the integral in
equation \ref{eq:marg}.
To do this, we need to specify priors $\Pi(\phi)$, but more importantly
we need to compute the likelihood $L(\vd | \theta, \phi)$.

We need, therefore, to make predictions for the 1D flux power spectrum
at a certain set of points, $P_{1D}(z,q)$, in velocity units, given a model
defined by ($\theta$,$\phi$).
We discuss the (nuisance) astrophysical parameters $\phi$ in more detail
later on, but for now we will assume that there are only two parameters:
an overall normalization of the logarithm of the mean flux $\ln{\bar F_0}$,
multiplying some fiducial redshift evolution, and an overall normalization
of the filtering length $k_{F 0}$ (associated to the smoothing of the gas),
also multiplying a fiducial redshift evolution. 


\subsection{Emulating a particular redshift}

For each redshfit $z_i$, we compute the corresponding value of the mean flux
($\bar F_i$) and filtering length ($k_{Fi}$).

Using the linear power parameters ($\theta$), and the fiducial
power spectrum (in velocity units), we are able to compute the expected
linear power for this particular model, in velocity units, at this redshift,
$\tilde P_i(q)$.

\AFR{This is one of the pieces that is still not crystal clear in my head.
Would we just take $\tilde P_L(z_\star,q)$, the linear power in velocity
units at the central redshift, and rescale it using the discussion around
equation \ref{eq:growth}?
If we did that, we would be missing the fact that the transformation between
comoving and velocity separations $a_v$ changes with redshift.
I guess one could use the fiducial model to compute this difference?
The alternative would be to have a 6th cosmological parameter describing
the difference in the change in the Hubble expansion around $z_\star$?}
\as{If DE is truly negligible for sensible models, then just don't
  worry about it and assumed EdS. If it makes small corrections, then
  the best course of action would be to also specify $d
  H^2/da=-3\Omega_m/a^4$ at $z=z_\star$.}


At this point, we can completely forget about the redshift $z_i$.

Instead, we will go to our simulation database, and ask:
is there any simulated flux power spectra that was computed from a snapshot
with similar linear power (in velocity units) $\tilde P_i(q)$, similar
mean flux $\bar F_i$ and similar filtering length $k_{Fi}$? \as{And
  similar $f$ for velocity effects? Again, only matters if non-EdS matters.}

If we had an extremely large number of simulations, we could just setup a
metric to find the closest snapshot, and directly read the flux power
from the snapshot.
Since we will have a sparse sampling of the parameter space, we will need
to do some interpolation between them. \as{This is a good way to think
  about this, yes.}
This interpolation is precisely the role of the \textit{emulator}.

Note that the internal metric used for the interpolation does not need to 
use the same parameters $\theta$ describing the linear power.

