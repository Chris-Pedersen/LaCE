\section{The emulator} \label{sec:emu}

We will go to our simulation database, and ask:
is there any simulated flux power spectra that was computed from a snapshot
with similar linear power (in velocity units) $\tilde P_i(q)$, similar
mean flux $\bar F_i$ and similar filtering length $k_{Fi}$? \as{And
  similar $f$ for velocity effects? Again, only matters if non-EdS matters.}
\pvm{Remember that much of non-EdS effects can still be accounted for by just
linear theory. } \as{But it changes growth, which in turn changes
velocity smoothing. So yes, linear power spectrum but also most likely
its time derivative (or equivalently velocity power spectrum)} \pvm{Thinking of neutrinos, I think it is really best to get away
from talking about $f(z)$, which is not well-defined when it is really 
$f(z,k)$. 
Remember that you can easily compute from CLASS the velocity power spectrum as
well as density, which gives you your leading order handle on changes in 
evolution... (arguably if you had to choose you'd probably want this instead
of density power for LyaF, but you don't have to choose...)}
\AFR{Yes, we could add linear velocity power instead of $f(z)$, and compute
from there any parameter we want to use internally.}

If we had an extremely large number of simulations, we could just setup a
metric to find the closest snapshot, and directly read the flux power
from the snapshot.
Since we will have a sparse sampling of the parameter space, we will need
to do some interpolation between them. \as{This is a good way to think
  about this, yes.}
This interpolation is precisely the role of the \textit{emulator}.

Note that the internal metric used for the interpolation does not need to 
use the same parameters $\theta$ describing the linear power.

