\section{Public likelihood code} \label{sec:like}

In order to properly discuss our choice of parameterization and simulation
setup, it is important to describe how we intend the end product,
the \textit{likelihood code}, to be used in a cosmological analysis.

Note that during this discussion we have ignored the differences between 
baryons and CDM, and always work with their weighted-averaged power. 


\subsection{User interface}

\as{I don't like the expression ``user interface'' here. Wouldn't
  likelihood parameterization be more appropriate?}
\AFR{Anze, this is an internal document, to make sure we are all (including
Chris, Hiranya, Andrew, Tom...) in the same page. 
The second part of this section is about the parameterization, this first 
part is really just about the interface... 
As Pat mentiones below, we can give an easy interface to the users, 
and under the hood we compress that into any set of parameters we want.}

The aim of the project is to provide a public likelihood package that others
can use to include \lya\ results in their cosmological parameter constraints.
The end-user does not need to know about the internal details, about the 
suite of simulations, the nuisance parameters or the interpolation scheme 
used in the emulator.
They do not need to know either whether internally we are describing the 
power in units of $\kms$, $\Mpc$ or $\hMpc$. 

There are different possible interfaces that we could setup, and probably 
we will want to provide more than one with different levels of complexity.
But we will start discussing a particular interface, where we will ask 
the user to provide for each cosmological model:
\begin{itemize}
 \item Linear matter power (baryons+CDM), as a function of redshift and 
  wavenumber, in units of $\iMpc$. 
  The redshift range should cover at least $2 < z < 5$, and the wavenumber
  range should cover at least $0.01 \iMpc < k < 10 \iMpc$. 
  For simplicity, we will restrict ourselves to models with a linear power 
  spectra that can be factorized ,in the range of redshift and scales 
  described above, in a power spectrum at a central redshift, 
  $P_L(z_\star=3,k)$, and a scale independent growth factor, $D(z)$.
  \as{If you are to compress flux measurement over X redshift bins
    into one linear power constrain, just ask for the linear power
    at that $z_{\rm pivot}$ and $f$ at that redshift. If you are going
    to compress into two $z$s, ask for those two $z$s and two $f$s}.
  \AFR{Yes, this is an option. But as Pat mentions below, this might not 
   be the easier option for some users. But when I mentioned above that we 
   could have more than one interface, this is what I had in mind.}
 \item Logarithmic growth rate, $f(z)$, that could be computed directly
  from the evolution of $D(z)$.
 \item Hubble parameter as a function of redshift, $H(z)$, over the same 
  redshift range.
 \item If we want the likelihood code to be able to fit 3D clustering as well,
  we would need as an input the angular diameter distance, $D_A(z)$, 
  over the full redshift range. \as{I would add $r_d$ and also $r_d$
  fiducial for generality}.
  \AFR{Yes, I forgot about $r_d$ here. Although we could fit it from the 
  linear power ourselves under the hood, to prevent some users to use EH98 
  and others using CAMB to compute it.} 
\end{itemize}

\pvm{I don't think you should skimp on what you want from the outside. E.g., 
in neutrino mass era it is starting to feel a little quaint to talk 
about $D(z)$, 
$f(z)$ -- both generally depend on $k$. At the same time, producing density 
and velocity power spectra as a function of k and z is just getting easier and 
easier as people get more comfortable with things like CLASS. 
(I mean, no I 
don't actually think $k$ dependence of $D$ and $f$ will ever matter, 
but it seems
just easier to ask for the linear $P(z,k)$ than worry about defining and
justifying these things
(e.g., I remember kind of
laughing at Alexey Makarov for producing $D(z)$ by running CMBfast, which 
seemed like overkill when I had a little numerical integrator code, but to him
this was actually easier than setting up that code...))
I wouldn't worry
too much about what you're going to ask for though. Focus on making a good 
emulator and then ask for what you need for that... 
}
\AFR{Great, that's all I wanted to hear. For now, let's assume this is a 
possible option, and we'll come back to it later on.} 

The user will also specify:
\begin{itemize}
 \item Data products to use:
  SDSS-I from \cite{McDonald2006},
  BOSS from \cite{Palanque-Delabrouille2013},
  HIRES/UVES from \cite{Viel2013},
  XQSO-100 from \cite{Irsic2017},
  HIRES/UVES from \cite{Walther2018a}.
  We should also allow the user to specify what redshifts to use from each
  dataset, since they are independent, and probably also allow the user
  to specify the scales to be used in the fit for each dataset.
 \item Extra analysis settings, like whether to allow for running in the
  linear power, differences in the linear growth, or differences in the BAO
  wiggles.
\end{itemize}

\AFR{It is not clear to me whether at this point the user would also be able
to set other settings of the likelihood or not, like the way we treat
contamination by DLAs or metals, resolution or noise corrections, or the
parameterization of the temperature-density relation in the simulations.}
\as{This is implementational detail, no need to worry about this now.}

\AFR{I believe the answer is that each of these analysis choices would be
a different likelihood object, and then the user can decided whether to
use the likelihood object where the DLA marginalization used the formula
from \cite{McDonald2005b} or whether to use the one that used the formula
from \cite{Rogers2018a}.
Similarly, every time we want to use a different prior for the nuisance
parameters (say we want to include measurements of mean flux or temperature)
we would need to recompute the marginalization, and provide a new likelihood
look-up table.}

The output from the \textit{likelihood code} will be:
\begin{itemize}
 \item A value of (log-)likelihood for each dataset, possibly a value for 
  each redshift bin. 
  Most users will only care about this.
 \item For the experts, we would also output the best-fit theoretical 
  prediction for each dataset, for that particular cosmological model.
  We would also provide the values of the nuisance parameters that correspond
  to the best fit model (mean flux, temperature-density relation, metal or DLA
  contamination...).
 \item We could also provide a random sample of theory lines that are above a 
  likelihood threshold for that particular model, exploring different 
  thermal histories and other nuisance parameters.
\end{itemize}

\AFR{Actually, I don't think at this point you can get any of the above.
The user of the final likelihood code will only get a total value for the
likelihood for the total dataset.
Even though one would compute the un-marginalized likelihood for each redshift
and each dataset separately, at the step of marginalizing over nuisance 
parameters (see section \ref{sec:emu}) you would need to include them 
all at the same time.}
\as{No, no. You don't marginalise over nuissance parameters inside
  your likelihood code. Some nuissance parameters will have particular
  degeneracies with the big picture cosmological parameters, and in
  addition you never know how clever the user's sampler is. It is ok
  to ask for linear power, $f$, $H(z)$, $D_A(z)$ and a set of
  nuissance parameters, which users doesn't need to understand except
  for valid ranges. You can add a small wrapper that does this
  marginalisation internally for those who don't care.}
\pvm{In principle I more or less agree with \anze\ about the nuisance 
parameters, the core level of the code should be written treating all 
parameters symmetrically and this should be kept accessible... although,
if you were, e.g., importance sampling a Planck chain you would effectively
return to needing the code to marginalize over nuisance parameters. On the 
other hand, I remember long ago in Alexey Makarov days we ended up retreating
from this idea because the generic MCMC was painfully slowed by marginalization
over these nuisances when we knew how to make it fast internally... Ugh - 
I'm going to try not to read anything but emulator section since I think this
discussion of final public likelihood code is getting way ahead of yourself... 
\smiley\
I'd focus on ``what do I need to do to get an emulator capable of any kind of
fit to a LyaF data set?" ...  }

\AFR{Nooooooo! I need to understand how the final package will be used, 
before I can think of emulators and simulations!}

\AFR{Pat, in your 2005 paper, you had $\chi^2$ look-up tables, that were only
a function of amplitude and slope (and may be running). 
I always thougth that this type of look-up table would be the end product of 
the analysis, with a nice wrapper around it with a public interface to 
translate cosmology to our cryptic parameters in the lookup table.
Isn't it the case? 
You made the look-up table public, and that table had already marginalized
over all nuisance parameters, right?}


\subsection{From cosmological model to likelihood parameters}

Under the hood, we will use more effective (and cryptic) parameters in our 
likelihood, to reduce the internal degeneracies between parameters.
There are many, many possibilities here, but we will start by discussing a 
possible setting. 

\subsubsection{Fiducial cosmological model}
 
We will choose a fiducial cosmological model, based on a recent Planck+BAO 
analysis, and use it to compute a fiducial linear power spectrum, $P_L^0(z,k)$, 
a fiducial Hubble expansion, $H^0(z)$, a fiducial growth rate $f^0(z)$...
All quantities with a superscript $^0$ will refer to the fiducial model.

\subsubsection{Linear power shape}

Since we have decided to use only models where the linear power spectrum 
can be factorized, we will describe its shape at the central redshift only, 
$z_\star=3$. 
In general we will use the subscript $_\star$ to refer to quantities that 
have been evaluated at $z_\star$, but in this sub-section we will ignore 
the redshift and assume that all power spectra are evaluated at $z_\star$.

We will compress the difference between the input power spectrum, $P_L(k)$, 
and the fiducial power spectrum, $P^0_L(k)$, into a handful of parameters. 
We start by fitting the fiducial power with a smooth function, 
$P^0_{nw}(k)$, using the \textit{no-wiggle} model from \cite{Eisenstein1998}.
We define the oscillatory (or \textit{wiggle}) component of the 
fiducial power, $W^0(k)$, as the ratio of these two powers:
\begin{equation}
 W^0(k) = \frac{P^0_L(k)}{P^0_{nw}(k)} ~.
\end{equation} 
\pvm{Are your boxes actually going to be big enough to worry about BAO? If 
so, I think you should *definitely* model that part by linear bias/more 
general PT... you can use that to pull out the feature essentially 
analytically. Never mind what I said about ``units of BAO"... this is surely 
the way to go since you don't want to be trying to interpolate between raw
sim power at that scale anyway...}

We will assume that the oscillatory component of the input model can be 
described with the oscillations in the fiducial model, shifted by the ratio 
of their sound horizons at the drag epoch ($r_d$):
\begin{equation}
 W(k) \sim W^0(\beta k)  
\qquad  
 \beta = \frac{r_d}{r^0_d} ~.
\end{equation}
We have decided to use $\beta$ and not $\alpha$, more common in BAO analyses,
because the latter includes a ratio of transformations from observable to 
commoving coordinates, and we do not need this at this point. 

Finally, we will model the differences in the smooth component with a smooth,
parameterized function:
\begin{equation}
 B(k) = \frac{P_{nw}(k)}{P_{nw}^0(k)} ~.
\end{equation}
There are different parameterizations possible, but for the rest of this 
discussion we will assume that we use three parameters: an amplitude, 
a slope, and a running of the slope, evaluated around $k_p = 1 \iMpc$. 

To summarize, we will describe the input linear power at $z_\star$ as:
\begin{equation} \label{eq:Pk_param}
 P_L(k) = B(k) ~ P_{nw}^0(k) ~ W^0(\beta k) ~.
\end{equation}

\subsubsection{Linear growth}

We will compute the difference of the input logarithmic growth rate with that 
in the fiducial cosmology, $\Delta f_\star = f(z_\star) - f^0(z_\star)$, 
evaluated at $z_\star=3$. 
We will approximate that the different growth rate at $z_\star$ is enough 
to compute the difference in linear growth at any redshift (within the range):
\begin{equation}\label{eq:growth}
 \frac{D(z)}{D^0(z)} = 1 + \Delta f_\star ~ \frac{\Delta z}{1 + z_\star} ~.
\end{equation} 
Note that in LCDM models, and at $2 < z < 5$, the differences in growth rate 
are typically less than 1\%, as shown in Figure \ref{fig:fz_Om}.

\subsubsection{Hubble expansion}

If we could observe the \lya\ power spectrum in comoving coordinates, that 
would be enough. 
However, we observe the power spectrum in observing coordinates, wavelengths
and angles, and a more natural choice is to use velocity units ($\kms$) for 
the clustering measurements. 
Indeed, all recent measurements of the 1D \lya\ power reported their results
in units of $\kms$, and we will assume the same in this discussion.

In general, we would need to use $H(z)$ from each model to compare 
measurements in $\kms$ with model preditions in $\Mpc$:
\begin{equation}
 q = \frac{1+z}{H(z)} ~ k = a_v k~,
\end{equation}
where we use $q$ to refer to wavenumbers in velocity units, and we have 
defined $a_v$ as the transformation from $\kms$ to $\Mpc$.
This would force us to add in the emulator some sort of Hubble 
parameter, either at $z=0$ ($h$) or at $z_\star=3$. 
However, as suggested by \cite{McDonald2005a}, it is possible to avoid this
burden if we describe our model (the linear power spectrum) already in 
units of $\kms$. \as{This confused the shit out of me until very
  recently. I vote for having power spectra in Mpc and also require
  $H(z_\star)$ and do this conversion internally -- it is trivial. }
\AFR{Yes, Anze, that was the entire point of the user interface. 
We ask the users to provide power in $\Mpc$, and we ask for $H(z)$, 
and we do the convertions under the hood.}

We claim that two models with different expansion histories $H(z)$, but the
same linear power in units of $\kms$, will have very similar \lya\ power
spectra, with small remaining differences being caused by astrophysical 
effects (different reionization history, different thermal history, different
mean flux...). 
And since we plan to marginalize over these to get the final cosmological 
constraints, we do not need to worry about these differences. 
For the rest of this discussion, we will assume that this is true.

\pvm{I agree with \anze\ that you should move to just using Mpc and asking for
$H(z)$, where, a clincher is in 3D you can't get
away from asking for at least $(H D_A)(z)$. The argument about km/s was always
that you should quote results as a linear power measurement in km/s at 
$z\sim 3$, *if you are going to quote results from LyaF $\sim$alone* -- the
goal there was to produce ``model independent" constraints that could be
propagated forward. But if 
you're going to be using flux power directly in global chains fitting 
cosmological parameters, which seems to be what you're talking about here, 
that argument doesn't apply -- you have the only relevant cosmological model 
on hand point-by-point. This does raise the question though, how much you 
want to commit to that approach, i.e., to primarily present a likelihood code
that does direct fits to flux power measurements on the fly, vs. boiling 
things down to linear power measurement (which you would again obviously 
present in observable coordinates). Certainly internally you want to be able
to do both, and the code to do linear boil-down should probably wrap the other
one, but there is a question where to spend more time cleaning and advertising. 
Arguments in favor of boil-down to linear could be speed (in old days, like 
I said, we could do both, but fit to flux power really slowed down global
fits, while my $\chi^2$ table gave identical results), and just... it is nice
to be able to show the more-or-less LyaF-model-independent cosmological thing
you claim to have measured, not just present a big black likelihood box. 
It helps, e.g., estimate how these constraints will affect new cosmological  
models. In any case, you can figure out what to advertise later. It seems like
first goal should be to produce $\chi^2$ contours for, e.g.,  
$\Delta^2,\neff(z=3, k=0.009 s/km)$,  
defined as deviations from a central model (i.e.,
effectively variations of $A_s$ and $n_s$ with other parameters fixed), 
to compare to past... this is clearly going to be *most of* what matters.} 


How does this affect the discussion above?

Let us use $\tilde P(q)$ to refer to power spectra in units of velocity:
\begin{equation}
 \tilde P(q) = a_v^3 ~ P(k= q / a_v) ~.
\end{equation}

We can then redo the whole discussion above, but using power spectra in 
velocity units:
\begin{align} 
 \tilde P^0_L(q) & = (a^0_v)^3 ~ P^0_L(q / a^0_v)         \nonumber \\
   & = (a^0_v)^3 ~ P^0_{nw}(q / a^0_v) ~ W^0(q / a^0_v)      \nonumber \\
   & = \tilde P^0_{nw}(q) ~ \tilde W^0(q)  ~,
\end{align}
where we have also defined $\tilde W^0(q) = W^0(q / a^0_v0$.

We can now define a term for the ratio of the smooth power, in velocity
units:
\begin{align}
 \tilde B(q) & = \frac{\tilde P_{nw}(q)}{\tilde P^0_{nw}(q)}    \nonumber \\
  & = \left(\frac{a_v}{a_v^0}\right)^3 
      \frac{P_{nw}(q / a_v)}{P^0_{nw}(q / a_v^0)} ~,
\end{align}
and we can finally write 
\begin{align}
 \tilde P_L(q) & = \tilde P_{nw}(q) ~ W(q / a_v)                \nonumber \\
  & = \tilde B(q) ~ \tilde P^0_{nw}(q) ~ \tilde W^0(\alpha q) ~,
\end{align}
where we have defined $\alpha$ as the ratio of sound horizons in units of 
velocity: 
\begin{equation}
 \alpha = \beta \frac{a_v^0}{a_v} = \frac{r_d ~ H_\star}{r_d^0 ~ H^0_\star}~.
\end{equation}

The cosmological model in the likelihood will then be described by a 
set of parameters $\theta$ describing the linear power spectrum, including:
\begin{itemize}
 \item Ratio of sound horizons in units of $\kms$, $\alpha$, where the 
conversion from $\Mpc$ to $\kms$ is computed at $z_\star=3$. 
This is the inverse of the usual definition of BAO $\alpha_\parallel$.
 \item Approximately 3 parameters describing the ratio of the smooth
  linear power at $z_\star=3$, in units of $\kms$.
 \item Difference of growth rates at $z_\star=3$, $\Delta f_\star$. 
\end{itemize}

We will ask the likelihood code: for these set of parameters $\theta$,
what is the likelihood of getting the measured power, after marginalizing
over all nuisance parameters (including mean flux and thermal history)?
Or in math, what we want is:
\begin{equation} \label{eq:marg}
 L(\vd | \theta)
  = \int d\phi ~ \Pi(\phi) ~ L(\vd | \theta, \phi) ~,
\end{equation}
where $\vd$ is the measured flux power spectrum, $\phi$ are the nuisance
parameters, and $\Pi(\phi)$ are the priors on the nuisance parameters.
\as{Again, marginalisation is for the user to do. The code should
  output total $\chi^2$ at this point and optionally data-points and
  theory predictions at this point include full gory of nuissance
  parameters. If you look at eg \texttt{cosmomc} it internally
  marginalises already over some 15 Planck parameters, it can do 15
  more for us.}
\AFR{Ok, this is the type of discussion I wanted to have. 
Is this really how it was done in SDSS-I? What was the look-up table for, then?
To make plots?}

This likelihood will have been evaluated at a grid of points in $\theta$,
and it will be stored as a look up table.
Evaluating the likelihood, once the lookup table has been computed,
should then be trivial and extremely fast.

