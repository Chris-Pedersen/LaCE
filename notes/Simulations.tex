\section{Simulations}
\label{sec:sims}

Even though we could always move to Gadget-3, we would like to start the 
project with MP-Gadget as our main code to evolve the initial conditions. 
We should do a comparison of the linear growth in both codes, and that should
help us make a final decision.

Our baseline plan is to run simulations with $N=1024^3$ particle per especies
(baryons and CDM) in boxes of roughly $100 \Mpc$.
This would give us a mean particle separation of roughly $100 \kpc$, or 
$70 \hkpc$, that should be enough to have a first decent emulator working.
\afr{This might not be enough to properly study the higher redhift bins in
HIRES/UVES, $z > 5$, where the gas is less smooth and it is harder to resolve
all the structure.
But I think it is better to err on this side than to have boxes that are too
small, since poor resolution is somewhat degenerated with different
reionization / filtering length that we want to marginalize over anyway.}

We discuss here the different post-processing of the snapshots, and how we
store the simulated power spectrum that the emulator will later use to make
predictions.


\subsection{Rescaling of the optical depth}

From each simulation we will get a serie of snapshots, outputs at different
redshifts. 

From each snapshot, we will extract \lya\ skewers, and use these to compute
their power spectra.
We will repeat this exercise for different rescalings of the optical depth,
i.e., we will multiply the optical depth in all cells by a constant factor
in the range $0.5 < A_\tau < 2.0$ (approximately), and for each value of
$A_\tau$ we will compute the transmitted flux fraction $F$, its mean value
(mean flux), and the power spectra of their fluctuations $\delta_F$.

Therefore, from each snapshot we will get a set of power spectra for different
values of $A_\tau$.
We could label the different power spectra by their associated value of
$A_\tau$, but we will label instead by their resulting value of the mean
flux $\bar F$.
\cite{Lukic2015} argued that the rescaling might introduce biases in the
1D power spectrum for values of $A_\tau$ very different than one.
However, their test compared two simulations with different thermal history
and different pressure, so it is difficult to tell whether the bias came
from the rescaling or from the different IGM physics.
It is true that collisional ionization could break the rescaling, but this type
of ionization should only be relevant on very high-density regions,
where we know that our simulations are not correct anyway
\footnote{So if we found that collisional ionizations mattered we would
have to consider using more realistic (and expensive!) simulations.}.

\afrhid{It would be great to repeat this exercise in two simulations that
have very similar thermal history but different mean flux, I will ask
Jose Onorbe for help (he is visiting UCL soon).
It is also possible that the test is clearer if we look at the 3D power,
where pressure only affects the high-k limit, and the scale independent
linera bias.}
\pvmhid{It seems useful to be sure to understand the physics you're talking
about here.
If I am remembering correctly, a change in ionizing background at the
output time will change the ionization rate exactly proportionally, which
will change the neutral density and therefor optical depth exactly
proportionally, in the
ionization equilibrium, fully ionized, only photo-ionization limit.
So to say the rescaling ``doesn't work" is to say you have a breakdown
somewhere in here. E.g., collisional ionization violates
``only photo-ionization", but generally only at very high density, where...
if this was what was going on you really need to find a way to desensitize
yourself to this high density, because surely you are getting it wrong in
other ways. Basically I think this is the story with all possibilities, i.e.,
there is no possibility that the solution to any imperfection in this scaling
would be to not do the rescaling, because there is *nothing you can simulate
reliably* that can violate it. Much more likely (than collisional ionization)
of course, like you say here, a
change someone finds when they rerun with different ionizing background is
from changing temperatures, which you are supposed to be marginalizing over.
}
\afrhid{Yes, you are right. I had a chat with Jose about this a couple of weeks
ago, and he mentioned collisional ionization as a possible reason for the
breakdown, but as you say we don't trust these high-density regions anyway.
It is more likely that the reason that \cite{Lukic2015} finds a "bias" is that
they have different thermal histories in the two simulations (they acknowledge
this is the case).
I will rewrite this section to make this more clear.}


\subsection{Rescaling of the temperature}

The Temperature-Density Relation (TDR) in the \lya\ forest can be reasonably
well described by a power law, 
\begin{equation}
 T(\rho) = T_0 \left(\frac{\rho}{\rho_0}\right)^{\gamma-1} ~,
\end{equation}
with a typical values for $\gamma$ between 1 and 1.6. 
If we use $\rho_0 = \bar \rho$, $T_0$ varies between 10,000 and 20,000K
\cite{Lukic2015}.

We can change the thermal history of the simulations by running the same box
with different \textit{TREECOOL} files, that contain the redshift evolution
of different heating and ionizing rates.

For each snapshot, we can fit their values of $T_0$ and $\gamma$, and use 
these to label the snapshot, instead of using the name of the TREECOOL, 
or the parameters that we have used to modify a given file 
(MP-Gadget can implement the recipes in \cite{Bolton2008} to modify TRECOOL
files by setting the parameters \textit{HeatAmplitude} and \textit{HeatSlope}).

Just like we did with the optical depth, we can rescale the temperatures in 
post-processing. 
This would capture correctly two different physical effects:
\begin{itemize}
 \item Thermal broadening: when extracting the skewers from the boxes, 
  the last step is to compute the redshift-space-distorted optical depth. 
  To do that, the temperature at each cell is used to decide the local 
  smoothing that we need to apply, what is known as the thermal broadening. 
  It is trivial to take the same optical depth skewer, and convolve it for
  different temperature rescalings.
 \item Recombination rates: the temperature also sets the recombination rate,
  $\alpha(T) \sim T^{-0.7}$. 
  We could recompute the recombination rate at each cell, 
  propagate this into a change in the neutral fraction (proportional to the 
  recombination rate), and finally to the optical depth (proportional to the 
  neutral fraction).
 \pvmhid{Note that these two are both essentially instantaneous, i.e., I don't 
think there is any physical possibility of them not using the same 
temperatures... (so I wouldn't mess with it, except maybe in some pedagogical
example)}
 \afrhid{Great, thanks.}
\end{itemize}

It does not matter whether we use different TREECOOL files or different
\textit{HeatAmplitude} and \textit{HeatSlope} parameters, at the end of the
day we will compute the actual values of ($T_0$,$\gamma$) in the snapshots
and store that information to described the power measured in the
post-processed snapshot.


\subsection{Pressure smoothing}

The small scale structure is suppressed on very small scales because of the
pressure in the gas.
As described in \cite{Hui1997,Gnedin1998}, the smoothing can be described
by a characteristic scale, $k_F(z)$, the \textit{filtering scale}, that is
an integral version of the Jeans length that depends also on the temperature
in the past.
It is important to distinguish the effect of pressure from the effect of
temperature, since one can have two snapshots with very similar temperature
but very different values of $k_F$ because of a different reionization
history.

\pvmhid{Which is of course not really the same temperature (unlike above), i.e.,
the functional derivative of $P(z)$ with respect to $T(z')$ through the effect
of pressure is zero at $z=z'$, while it is a delta function through thermal
broadening and recombination rate... I know you know this, this is just a
different way to say it... if all pressure came from temperature in observable
range I'd say this is a little bit of a nitpicky distinction, but the
contribution from pressure at early times sort of fundamentally decouples the
two uses of temperature...}
\afrhid{Agreed.}

\afrhid{Unfortunately it is very difficult to add smoothing in post-processing,
effectively lowering the value of $k_F$ in the simulation by hand.
Of course, it would never be possible to reduce the smoothing, and to do that
one would need to run a simulation with an earlier redshift of reionization.}

Note that it is not trivial to measure $k_F$ from a snapshot.
One could be tempted to measure the gas density power spectrum and fit an
effective Gaussian cut-off, but this power spectrum is completely dominanted
by the non-linear regime (high densities) that we do not trust in our 
simulations.
In several papers by Onorbe/Lukic/Hennawi 
\cite{Lukic2015,Onorbe2016,Walther2018a,Walther2018b} they fit $k_F$ from
the power spectrum of $e^{-\tau}$ in real space, without adding redshift
space distortions.
An alternative would be to use the information about the temperature in
previous snapshots to compute $k_F$ using equation $8$ in \cite{Gnedin1998}.

\afr{I wonder whether this would require storing even more snapshots,
increasing even further the total amount of disk space required.
It is possible that we can compute the temperature on the fly, or write
only a sub-sample of particles for "intermediate" snapshots.}

\afrhid{We could also ask Jose Onorbe for help to setup this type of test. 
We would also need to discuss the best way to measure $k_F$ in the snapshots:
fit a Gaussian kernel in the power spectrum of $F_{\rm real}$, where no 
redshift-space distortions have been included? I believe that is what is 
used in all papers by Hennawi / Lukic / Onorbe.}

\ashid{I remember having massive trouble with this when Nishi was still
  around. I naively imagined that I would just plot the 3D baryon
  power spectrum, see supression by eye and call $k$ at the midpoint of
  supression the smoothing. No cigar -- the baryon power spectrum is
  dominated by shot-noise of baryons in halos, you really need to be a
  bit clever about this. Perhaps this has been solved.}

\pvmhid{What \anze\ says here sounds right -- power spectra that don't account
for $e^{-\tau}$ effect (desensitizing to high density) are never very 
relevant, because always dominated by high density (in non-linear regime). 
But their method *does* account for this at some level -- it is why they 
define ``$F_{real}$". 
But what I had in mind (and more or less did in the past) is quantifying
your TDR(z) at relevant densities by something like median fit to gas at 
those densities, and then 
literally computing Gnedin \& Hui analytic $k_F$ based on this TDR(z).  
You aren't trying to compute a filtering length you can interpret literally, 
just some number that your emulator can map into a quantified level of 
smoothing in flux power (as a bonus though, since the relation between $k_F$
and past temperature is explicitely defined here, you *can* relate fit 
results for this $k_F$ back to past temperature possibilities, without
needing sims to make *that* mapping, as you do if you define $k_F$ by some
complicated statistics of simulation outputs). None of these things are going
to be perfect -- you just need something for a first round, after which you 
look for deviations and try to figure out what do do about them 
(of course, you also need to worry about winds from galaxies, etc.).} 


\subsection{Labelling the snapshots}

In the linear regime, and for a single specie, the growth of structure is
scale independent, and it can be described by the growth factor $D(z)$.
%\begin{equation}
% P_L(z,k) = \frac{D^2(z)}{D^2(z_0)} ~ P_L(z_0,k) ~.
%\end{equation}
If the \lya\ power spectra depended only on the linear power spectrum, this
would suggest that there would be a complete degeneracy between changing
the overall amplitude of the linear power spectrum and changing the redshift
at which we ouput the snapshot.
Therefore, we could use the amplitude of the linear power at a given snapshot
to label it, and we could potentially use different snapshots of the same
simulation to study models with different amplitudes of the linear power.

\afrhid{How do we measure the linear power in the snapshot?
I could see three options (in order of my preference):
predict it using the power measured in the initial conditions, and the relative
growth as computed from CAMB/CLASS;
measure the density power in the snapshot, and fit the growth factor from
the low-k part;
run a very cheap simulation without hydro and a very low value of $A_s$, to
compute the actual linear power in the simulation (that might sadly differ
from the predicted by CAMB/CLASS because of issues in the linear growht).}
\ashid{By far preferrable, in fact the only way that wouldn't look dodgy
would be not to measure it at all \smiley . It is a known quantity give
linear codes \smiley .}
\afrhid{Pat always claims that measured power is better than predicted power,
given cosmic variance in the box.
We shouldn't care about the mean power over infinite number of realizations,
it is more useful to talk about the actual power than went into the simulation,
and that is why we could measure linear power in the initial
conditions.}
\ashid{Ok, but then you have the IC power spectrum that you
can multiply by the correct transfer function.}

To sum up, each snapshot will be used to generate multiple simulated
fields in post-process (rescaling optical depth, temperature...),
and we will compute the power spectrum for each of the simulated fields.
We will also compute (using CAMB/CLASS) the predited linear power for the
snapshot (in velocity units) and its logarithmic growth rate, and will measure
from the snapshots other IGM properties like the mean flux $\bar F$, the TDR
parameters ($T_0$ and $\gamma$) and the filtering scale $k_F$.
\ashid{Perhaps you need $f$ again.}
\afrhid{Yes, good point.}

The emulator will have a list of all the models for which we have simulated
power spectra, and it will specify a metric quantifying the separation
between two models.
Every time we ask for a the prediction for a given model (specified by
$\bar F,~T_0,~\gamma,~k_F,~\tilde P(q),~f$), the emulator
will use the nearest points and (somehow) interpolate between these.

Redshift will NOT be a label describing the simulated field, and neither
will be the TREECOOL file or the redshift of reionization.
Since we will define the linear power in units of $\kms$, we will not 
need to include the Hubble parameter at the box as a label.
We will not care about any other cosmological parameter in the box
either.

\pvmhid{You probably want to label it in Mpc, e.g., for mode consistency
reasons discussed above, 
but you still don't need to record Hubble parameter, you just need to record
temperature and $k_F$ also in Mpc... This probably relies on the (I believe
good) approximation that the recombination coefficient is a power law in 
temperature -- you should think about it... }

