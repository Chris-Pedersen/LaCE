\section{Introduction} 

We will discuss all the steps involved in extracting cosmological information
from the small scale \lya\ forest power spectrum, from measuring the
\lya\ (or flux) power spectrum to presenting a module that can be used
by others to include our results in their cosmological constraints.

The goal is to start an internal discussion within the different Co-Is
of the DiRAC proposal, and make sure that we are on the same page before
starting to write the science case. 
The idea is not to define the ultimate setting, but rather to present a
first possible setting, that we can use as a baseline when discussing
possible improvements.

Most of the discussion presented here is in the context of the 1D flux power
spectrum, but it should be relatively easy to extend the formalism to 3D.
This will be particularly important in the context of DESI, since the
information content will be dominated by 3D correlations 
\cite{Font-Ribera2014,Font-Ribera2018}.

We start in section \ref{sec:over} with an overview of the different steps 
involved in a cosmological inference from the \lya\ power spectrum, and we 
continue in \ref{sec:like} with a description of the likelihood code, 
its user interface, and its internal parameterization. 
We describe the link between the likelihood and the \textit{emulator} in
section \ref{sec:emu}.
In section \ref{sec:sims} we discuss the simulations used in the emulator, 
and the post-processing of the snapshots. 
We finish in section \ref{sec:fore} with a discussion on how the same
simulations could be used to compute Fisher forecasts for different \lya\
statistics.
 
