\section{Reconstructing the linear power from the parameters} \label{app:P_L}

In section \ref{sec:like} we discussed how we compress the cosmological
information provided by the user to a handful of $\theta$ parameters
describing the linear power spectrum over the relevant redshift and scales.
This reduced set of parameters $\theta$ will be the ones used in the
compressed (or marginalized) likelihood.

In order to compute the marginalized likelihood we need to use the default
cosmological model ($P^0(z,k)$, $P_\theta^0(z,k)$ and $H(z)$) and the set
of $\theta$ parameters to compute the corresponding linear power at a
particular redshift $z_i$ and in velocity units, $\tilde P_i(q)$, as well
as the growth rate in that redshift $f_i$.
In this section I will describe in detail how we do this.

\begin{itemize}
 \item Using the linear power in the fiducial model $P_\star^0(k)$ and the
  Hubble parameter $H^0_\star$, both evaluated at $z_\star=3$, we compute
  the linear power spectrum in velocity units for the fiducial model,
  $\tilde P^0_\star(q)$.
 \item Using the shape parameters $\beta$, we compute the linear power
  spectrum in velocity units for the input model, $\tilde P_\star(q)$.
 \item Using the Hubble parameter in the fiducial cosmology $H^0_\star$
  we convert this to a linear power spectrum in comoving coordinates,
  $P_\star(k)$.
 \item Using the linear growth in the fiducial cosmology, $f_0(z)$, we
  translate that to the desired redshift, $P(z_i,k)$.
  If we had $\Delta f_\star$ as a free parameter, we would use that to
  slightly modify this redshift evolution.
 \item We finally use the Hubble expansion in the fiducial cosmology,
  $H^0(z_i)$, to convert the linear power in velocity units at the desired
  redshift, $\tilde P(z_i,q)$.
  If we had an extra parameter describing the different expansion history,
  we would use it here to slightly modified this last step.
\end{itemize}

Another way to look at it, is that we will take the linear power at 
$z_\star=3$, in velocity units, and try to find a modification of the
primordial power in the fiducial model that matches the power.
For instance, by trying to find different values of ($A_s$,$n_s$ and 
$\alpha_S$) with which the fiducial model can be modified to match the
input power. 
We would then use call CAMB/CLASS with the fiducial model cosmology, but
this modified primordial power, to make predictions for the linear power
at any redshift.
This, of course, is for cases where we ignore differences in $f(z)$ or
$H(z)$, but these could easily be introduced as well.

\AFR{I actually prefer the itemized description, but it is a matter of taste.}

\afhid{But I don't think I've got an answer to my question. 
We define a model, by choosing $\theta$ (linear power at $z_\star=3$, 
in $\kms$, and may be $f_\star$) and by choosing $\phi$ ($\bar F$, $k_F$...);
We then ask the emulator to give us the prediction for flux power spectrum
at $z=4$, and the emulator needs to figure out how to translate this model
($\theta$,$\phi$) to the parameters describing the simulations outputs, to 
figure out which one to use (imagine we have infinit simulations);
For the IGM parameters it is easy, we have a way to use $\phi$ parameters
and the fiducial evolution, and turn that into a prediction for 
$\bar F_i$ and $k_{F~i}$;
However, we also need a way to translate the $\theta$ parameters, and the 
fiducial cosmological model, to a linear power at $z_i=4$, in $\kms$. 
How do we do this? 
If we had still access to the full $P(z,k)$ and $H(z)$ that entered the 
user interface, that would be trivial. 
But we threw that information away because we claimed that the only thing 
that matters (the only parameters in our final likelihood) were there 
$\theta$ parameters.
So we need to be able to reconstruct any linear power from these, so that 
then we can look at the simulations and try to look for a "snapshot" 
(more precisely, one of the multiple reproccessed snapshots) that had 
this particular power spectrum.
I'm not sure this is any clearer... 
}
\pvmhid{*For the emulator*, who says you need to ``define a model" by choosing 
power at $z_\star=3$? Where by ``emulator" I mean this thing that takes 
some kind of relatively easy to compute quantities and produces what it thinks
would be simulation results given them. Maybe it is easiest to think of it 
by sort of back-propagation: you have a 1D flux power spectrum measurement at 
$z=4$, you need a prediction for it, what does your emulator need to know to 
predict it? I'd say at first approximation it needs to know $P(k,z=4)$, in 
km/s. So the input to the emulator is $P(k,z=4)$ in km/s, period, end of 
story for emulator -- it has a hard enough job doing this well, it doesn't
need to worry about where this $P(k in km/s,z=4)$ came from. }
\afhid{Yes, I agree. I realize now that I was using the word \textit{emulator}
in the wrong way, including what you call bellow "code to make a $\chi^2$ 
table".}
\pvmhid{If you're asking
like ``how would I use this emulator to make a $\chi^2$ table of final results 
for $\Delta^2,\neff(z=3,k=0.009 s/km)$", worrying about broader z (and k) 
dependence,
I think you just pick the current best cosmological model as fiducial, and
define $\Delta^2,\neff(z=3,k=0.009 s/km)$ effectively as variations of 
$A_s$ and $n_s$ -- this gives you your $P(k in km/s,z=4)$ to feed the 
emulator, but it is really a completely separate thing from the emulator. }
\afhid{Just to be sure. $\theta$ parameters describe the different shape of the 
linear power, in $\kms$, at $z_\star=3$. 
I can try to use these, and $H^0(z_\star)$ from the fiducial model, to compute 
the equivalent linear power in comoving coordinates.
Then I can translate that to a different redshift using the linear growth
of the fiducial model (may be corrected by difference in $f_\star$), and 
then use the Hubble parameter of the fiducial model at the redshift, 
$H^0(z_i=4)$ to compute the final power we needed to talk to the simulations.
Correct? The only thing this could break is if the redshift evolution of 
$H(z)$ and $H^0(z)$ were very different, but that should not be the case for
most models, and if it was we could add an extra parameter to take care of 
this. Did I understand it?}
\pvmhid{In practice, I think you want to use $\Delta^2,\neff(z=3,k=0.009 s/km)$
to fix $A_s$ and $n_s$ in your fiducial model (i.e., the rest of it is fixed,
but they are varied), then you compute linear power at whatever redshift you
want directly from CLASS. I'm sure this is equivalent to what you are 
saying about scalings, and I wouldn't even object if you actually did those
scalings, but you should understand that they are a shortcut for this 
$A_s, n_s$ setting $\rightarrow$ CLASS run, not anything else. Of course, 
if $H(z)$ or something deviates enough from your fiducial model this 
single-amp-slope
compression won't be good enough, but that is a different issue (an extended
table would add running of spectral index and arbitrary power law modification
of growth factor, allowing you to do fits testing consistency of those 
things... and I should say, with improved precision maybe deviations within 
realistic models will be big enough to matter... I'm not saying they won't,
just that you should have a baseline idea and then think about whether you
need to expand).  }
\pvmhid{Of course, maybe you want to fit for growth deviations, and this gives
you a different way of getting $P(k in km/s,z=4)$, or maybe you are doing a big
global MCMC chain... the emulator who's job is to predict 
1D flux power at z=4 doesn't want to know what you are doing globally, it 
just wants to know $P(k in km/s,z=4)$... (I've been writing 
$P(k in km/s,z=4)$ because I carefully wrote 1D power and it is shorter than
writing ``$P(k in Mpc,z=4)$ and $H(z=4)$", which I think we agree is probably
how things should really go for pedagogical reasons, and 
add $D_A(z=4)$ for 3D.) }
\afhid{Ok, I think we are getting closer. I think part of the confusion was 
my poor use of the word \textit{emulator}, and the other part of the confusion
is that I always wanted to focus on the "look-up table" version, and not on 
the cosmomc version.}

