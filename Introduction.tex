\section{Introduction} 

Discuss here the unique window opened by Lyman-$\alpha$ (\lya) forest 
clustering, to study the linear power spectrum on small scales and redshifts 
higher than those available from galaxy surveys.

Discuss the role of hydrodynamic simulations in these studies, and the need 
for an emulator.

Discuss the importance of chosing the right parameterization in the emulator, 
since some cosmological parameters are not well measured by the \lya\ forest.
In particular, mention neutrino-mass degeneracy and the reasons to not use 
$\sigma_8$ defined at redshift zero.

Mention that even though this was studied already in \cite{McDonald2005a}, 
recent papers have ignored this issue \cite{Palanque-Delabrouille2015,
Yeche2017}. 
This will be relevant for future analyses, specially from DESI.

Even though past analyses have focused on the \textit{1D} power spectrum, 
the 3D power spectrum can also be measured \cite{Font-Ribera2018}, 
and it contains most of the information available from future surveys 
like DESI \cite{Font-Ribera2014}. 

\AFR{This should be updated as we have now new sections}

In this paper we take another look at this topic, using modern hydrodynamic 
simulations and explicitely showing the accuracy of some of the approximations.
We start in section \ref{sec:over} with an overview of the different steps 
involved in a cosmological inference from the \lya\ power spectrum, and we 
continue in \ref{sec:like} with a description of the likelihood code, 
its user interface, and its relation to the \textit{emulator} in section
\ref{sec:emu}.
In section \ref{sec:sims} we discuss the simulations used in the emulator, 
and the post-processing of the snapshots. 
 
