\section{Cosmological analyses with the \lya\ forest power spectrum}
\label{sec:over}

In this section we present an overview of the different aspects involved in 
a cosmological analysis of the small scale clustering from the \lya\ forest
power spectrum.

\begin{itemize}
 \item Measurement of the flux power spectrum: calibrate the quasar spectra, 
  fit the quasar continua, measure 2-point functions, covariance matrices 
  and possible contaminants.
 \item Infer the linear power spectrum: using hydrodynamical simulations, 
  build an emulator to translate the measured flux power to constraints 
  on the linear power spectrum of density at the redshift of the measurement
  (z ~ 3).
 \item Cosmological constraints: combine the inferred linear power spectrum
  with external datasets (primarily CMB studies) to constraint the parameters
  of a particular cosmological model.
\end{itemize}

Recently some studies have merged the last two steps, but we will argue that
this is not a good idea. 

Discuss here how does the measurement look like (redshift bins, k-range, 
typically in units of km/s...).

Discuss here the main difficulties in the emulator, including thermal history, 
mean flux, but also rescaling of optical depth. 

Discuss how the flux power mostly measures amplitude and slope around k=1, and 
it is only in combination with CMB that one can break degeneracies and 
constraint multiple cosmological parameters, including neutrino mass. 

